%%%%%%%%%%%%%%%%%%%%%%%%%%%%%%%%%%%%%%%%%%%%%%%%%%%%%%%%%%%%%%%
% P1 Thyroid
% David Miguel Lozano
% Javier Martínez Ribera
% Universidad de Burgos - Septiembre 2016
%%%%%%%%%%%%%%%%%%%%%%%%%%%%%%%%%%%%%%%%%%%%%%%%%%%%%%%%%%%%%%%

% -------------------------------------------------------------
% Preamble
% -------------------------------------------------------------
\documentclass[a4paper,12pt,titlepage]{article}

% -------------------------------------------------------------
% Packages
% -------------------------------------------------------------
\usepackage[utf8]{inputenc}	% Unicode support
\usepackage[T1]{fontenc}		% Font encoding
\usepackage[english]{babel}	% Languaje
\usepackage{lmodern}	% Typeface 
\usepackage{graphicx}	% Add pictures
\usepackage{hyperref}	% Add a link to index entries
\usepackage{amsmath}	% Advanced math typesetting
\usepackage{amsfonts}	% Mathematical formulas
\usepackage{amssymb}	% Extended symbol collection
\usepackage{listings}	% Code formatting and highlighting
\usepackage{xcolor}		% Color package
\usepackage{enumitem}	% Customizing lists
\usepackage{parskip}	% Paragraph styles
\usepackage[numbers,sort]{natbib}	% Bibliography management 

% -------------------------------------------------------------
% Configuration
% -------------------------------------------------------------
% Images path
\graphicspath{ {images/} }
% Hyperlinks coloring
\hypersetup{
	colorlinks,
	linkcolor={green!40!black},
	citecolor={blue!50!black},
	urlcolor={blue!80!black}
}
% Define HRule
\newcommand{\HRule}[1]{\rule{\linewidth}{#1}}

\begin{document}
% -------------------------------------------------------------
% Cover
% -------------------------------------------------------------
\author{David Miguel Lozano \ Javier Martínez Riberas}
\title{P1 Thyroid}
\date{23-09-2016}

\begin{titlepage}
	\centering
	\includegraphics[width=0.2\textwidth]{images/pw-logo.png}\par
	\vspace{0.3cm}
	{\scshape\LARGE Universidad de Burgos \par}
	\vfill
	{\scshape\Large Computación Neuronal y Evolutiva \par}
	\HRule{2pt}
	{\huge\bfseries Thyroid \par}
	\HRule{2pt}
	\\ [0.5cm]
	{Diseñar y entrenar una red neuronal para clasificar pacientes en tres grupos clínicos (sanos, con hipertiroidismo o con hipertiroidismo).}
	\vfill
	Student:\par
	{\Large\scshape David Miguel Lozano \\ Javier Martínez Riberas \par}
	\vfill
	Profesor de la asignatura:\par
	Álvaro \textsc{Herrero Cosío}
	\vfill
	{\large 1º semestre 2016 \par}
\end{titlepage}

% -------------------------------------------------------------
% Contents
% -------------------------------------------------------------
\newpage
\tableofcontents
\begin{appendix}
  \listoffigures
  %\listoftables
\end{appendix}

% -------------------------------------------------------------
% Body
% -------------------------------------------------------------
\newpage

\section{Introduction}

El objetivo de la práctica es diseñar y entrenar una red neuronal para la clasificación de pacientes en tres grupos dependiendo de su glándula tiroides. Se trata de un problema de reconocimiento de patrones en el que dados unos datos de entrada la red neuronal será capaz de asignarlos a una clase. Haremos uso del dataset de ejemplo "Thyroid" que provée Matlab.

Realizaremos un estudio sobre el ajuste en el número de neuronas, ejecutando varios entrenamientos con la red y modificando el valor de este parámetro. Y discutiremos qué configuración nos proporciona los mejores resultados. Por último, explotaremos la red y analizaremos las salidas devueltas.

\section{Motivación}

La motivación que nos llevó a elegir el tema del trabajo fue el que ambos poseíamos algún familiar que sufría alguna alteración en la glándula Tiroides. Por otro lado,  el gran desarrollo que están teniendo en los últimos años las redes neuronales aplicadas a la medicina nos desató interés en adentrarnos en este tema.

\section{Descripción del conjunto de datos}

El conjunto de datos utilizados los hemos obtenido del dataset de ejemplo "Thyroid" que provée Matlab. Los datos provienen del UCI Machine Learning Repository. \citep{Asuncion+Newman:2007} 

El dataset contiene datos de 7200 pacientes agrupados en dos matrices:

\begin{itemize}[noitemsep]
	\item thyroidInputs: matriz de 21x7200 con los datos de los 7200 pacientes caracterizados por 15 atributos binarios y 6 atributos continuos.
	\item thyroidTargets: matriz de 3x7200 en donde se asocia un vector de tres clases a cada paciente. En este vector se define a cuál de las tres clases pertenece el paciente.
\end{itemize}

Las tres clases que contiene el dataset son:

\begin{enumerate}[noitemsep]
	\item Paciente sano.
	\item Paciente con hipertiroidismo.
	\item Paciente con hipotiroidismo
\end{enumerate}

\section{Descripción del procedimiento}

asdfasdf

\begin{figure}[!ht]
	\centering
	\label{fig:idea-enc}
	\includegraphics[width=\textwidth]{pw-logo.png}
	\caption{IDEA decryption subkeys}
\end{figure}

\section{Estudio sobre el número de neuronas}

asdfasdf

\section{Mejor resultado obtenido}

asdfasdf

\section{Explotación de la red}

asdfasdf

% -------------------------------------------------------------
% Bibliography
% -------------------------------------------------------------
\newpage
\bibliography{citations}
\bibliographystyle{plainnat}

\end{document}