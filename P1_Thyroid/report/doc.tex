%%%%%%%%%%%%%%%%%%%%%%%%%%%%%%%%%%%%%%%%%%%%%%%%%%%%%%%%%%%%%%%
% P1 Thyroid
% David Miguel Lozano
% Javier Martínez Ribera
% Universidad de Burgos - Septiembre 2016
%%%%%%%%%%%%%%%%%%%%%%%%%%%%%%%%%%%%%%%%%%%%%%%%%%%%%%%%%%%%%%%

% -------------------------------------------------------------
% Preamble
% -------------------------------------------------------------
\documentclass[a4paper,12pt,titlepage]{article}

% -------------------------------------------------------------
% Packages
% -------------------------------------------------------------
\usepackage[utf8]{inputenc}	% Unicode support
\usepackage[T1]{fontenc}		% Font encoding
\usepackage[english]{babel}	% Languaje
\usepackage{lmodern}	% Typeface 
\usepackage{graphicx}	% Add pictures
\usepackage{hyperref}	% Add a link to index entries
\usepackage{amsmath}	% Advanced math typesetting
\usepackage{amsfonts}	% Mathematical formulas
\usepackage{amssymb}	% Extended symbol collection
\usepackage{listings}	% Code formatting and highlighting
\usepackage{xcolor}		% Color package
\usepackage{enumitem}	% Customizing lists
\usepackage{parskip}	% Paragraph styles
\usepackage[numbers,sort]{natbib}	% Bibliography management 

% -------------------------------------------------------------
% Configuration
% -------------------------------------------------------------
% Images path
\graphicspath{ {images/} }
% Hyperlinks coloring
\hypersetup{
	colorlinks,
	linkcolor={green!40!black},
	citecolor={blue!50!black},
	urlcolor={blue!80!black}
}
% Define HRule
\newcommand{\HRule}[1]{\rule{\linewidth}{#1}}

\begin{document}
% -------------------------------------------------------------
% Cover
% -------------------------------------------------------------
\author{David Miguel Lozano \ Javier Martínez Riberas}
\title{P1 Thyroid}
\date{23-09-2016}

\begin{titlepage}
	\centering
	\includegraphics[width=0.2\textwidth]{images/pw-logo.png}\par
	\vspace{0.3cm}
	{\scshape\LARGE Universidad de Burgos \par}
	\vfill
	{\scshape\Large Computación Neuronal y Evolutiva \par}
	\HRule{2pt}
	{\huge\bfseries Thyroid \par}
	\HRule{2pt}
	\\ [0.5cm]
	{Diseñar y entrenar una red neuronal para clasificar pacientes en tres grupos clínicos (normales, con hipertiroidismo o con hipertiroidismo.}
	\vfill
	Student:\par
	{\Large\scshape David Miguel Lozano \\ Javier Martínez Riberas \par}
	\vfill
	Profesor de la asignatura:\par
	Álvaro \textsc{Herrero Cosío}
	\vfill
	{\large 1º semestre 2016 \par}
\end{titlepage}

% -------------------------------------------------------------
% Contents
% -------------------------------------------------------------
\newpage
\tableofcontents
\begin{appendix}
  \listoffigures
  %\listoftables
\end{appendix}

% -------------------------------------------------------------
% Body
% -------------------------------------------------------------
\newpage

\section{Introduction}

The purpose of this report is to introduce the International Data Encryption Algorithm (IDEA) and describe the implementation done.

"International Data Encryption Algorithm (IDEA), originally called Improved Proposed Encryption Standard (IPES), is a symmetric-key block cipher designed by James Massey of ETH Zurich and Xuejia Lai and was first described in 1991. The algorithm was intended as a replacement for the Data Encryption Standard (DES). IDEA is a minor revision of an earlier cipher, Proposed Encryption Standard (PES)."  \citep{wiki:idea}

\section{Motivación}

asdfasdf

\section{Descripción del conjunto de datos}

asdfasdf

\section{Estudio sobre el número de neuronas}

asdfasdf

\section{Mejor resultado obtenido}

asdfasdf

\section{}


% -------------------------------------------------------------
% Bibliography
% -------------------------------------------------------------
%\newpage
%\bibliography{citations}
%\bibliographystyle{plainnat}

\end{document}