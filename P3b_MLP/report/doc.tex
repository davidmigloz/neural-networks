%%%%%%%%%%%%%%%%%%%%%%%%%%%%%%%%%%%%%%%%%%%%%%%%%%%%%%%%%%%%%%%
% P3b MLP
% David Miguel Lozano
% Javier Martínez Riberas
% Universidad de Burgos - Noviembre 2016
%%%%%%%%%%%%%%%%%%%%%%%%%%%%%%%%%%%%%%%%%%%%%%%%%%%%%%%%%%%%%%%

% -------------------------------------------------------------
% Preamble
% -------------------------------------------------------------
\documentclass[a4paper,12pt,titlepage]{article}

% -------------------------------------------------------------
% Packages
% -------------------------------------------------------------
\usepackage[utf8]{inputenc}	% Unicode support
\usepackage[T1]{fontenc}		% Font encoding
\usepackage[spanish]{babel}	% Languaje
\usepackage{lmodern}	% Typeface 
\usepackage{textcomp} % Special symbols
\usepackage{graphicx}	% Add pictures
\usepackage{pgfplots} % Graphs and charts
\usepackage{hyperref}	% Add a link to index entries
\usepackage{amsmath}	% Advanced math typesetting
\usepackage{amsfonts}	% Mathematical formulas
\usepackage{amssymb}	% Extended symbol collection
\usepackage{listings}	% Code formatting and highlighting
\usepackage{xcolor}		% Color package
\usepackage{enumitem}	% Customizing lists
\usepackage{parskip}	% Paragraph styles
\usepackage[a4paper]{geometry} 		% Margins
\usepackage[numbers,sort]{natbib}	% Bibliography management
\usepackage{booktabs}							% Tables

% -------------------------------------------------------------
% Configuration
% -------------------------------------------------------------
% Images path
\graphicspath{ {img/} }
% Graphs configuration 
\pgfplotsset{width=\textwidth,compat=1.9}
% Hyperlinks coloring
\hypersetup{
	colorlinks,
	linkcolor={green!40!black},
	citecolor={blue!50!black},
	urlcolor={blue!80!black}
}
% Define HRule
\newcommand{\HRule}[1]{\rule{\linewidth}{#1}}
% Define listings styles
\definecolor{codebg}{HTML}{EEEEEE}
\definecolor{codeframe}{HTML}{CCCCCC}
\definecolor{comments}{HTML}{009900}
\lstset{
  language=Matlab, 								% Programming language 
  backgroundcolor=\color{codebg},	% Background color
  frame=single, 									% Add frame around code
	framesep=10pt,									% Padding
	rulecolor=\color{codeframe},		% Don't change frame color
	upquote=true,										%
	breakatwhitespace=true,					% Break line only in spaces
	keepspaces=true,								% Keep indentation
	tabsize=2,											% Tab size
	title=\lstname, 								% Show filename as caption
	basicstyle=\ttfamily, 					% Size and font
  keywordstyle=\color{black}\ttfamily,
	commentstyle=\color{comments},	% Color of comments
  morecomment=[l][\color{magenta}]{\#}
}
% Define style table
\setlength{\heavyrulewidth}{1.5pt}
\setlength{\abovetopsep}{4pt}
\begin{document}
% -------------------------------------------------------------
% Cover
% -------------------------------------------------------------
\author{David Miguel Lozano \ Javier Martínez Riberas}
\title{P3 Multilayer Perceptron (MLP)}
\date{07-11-2016}

\begin{titlepage}
	\centering
	\includegraphics[width=0.16\textwidth]{ubu-logo.png}\par
	\vspace{0.3cm}
	{\scshape\LARGE Universidad de Burgos \par}
	\vfill
	{\scshape\Large Computación Neuronal y Evolutiva \par}
	\HRule{2pt}
	{\huge\bfseries P3: Multilayer Perceptron (MLP) \par}
	\HRule{2pt}
	\\ [0.5cm]
	{Diseñar y entrenar distintos Perceptron Multicapa (MLP), con el objetivo de hacer una comparativa respecto al rendimiento de estos para la misma tarea y aplicación estudiada en P1\_Thyroid.}
	\vfill
	Estudiantes:\par
	{\Large\scshape David Miguel Lozano \\ Javier Martínez Riberas \par}
	\vfill
	Profesor de la asignatura:\par
	\textsc{Álvaro Herrero Cosío}
	\vfill
	{\large 1º semestre 2016 \par}
\end{titlepage}

% -------------------------------------------------------------
% Contents
% -------------------------------------------------------------
\newpage
\tableofcontents
\begin{appendix}
  %\listoffigures
  %\listoftables
\end{appendix}

% -------------------------------------------------------------
% Body
% -------------------------------------------------------------
\newpage

\section{Introduction}

El objetivo de la práctica es diseñar y entrenar distintos Perceptron Multicapa (MLP), con el objetivo de hacer una comparativa respecto al rendimiento de estos para la misma tarea y aplicación estudiada en P1\_Thyroid (clasificación de patrones: tiroides).

Se realizará un estudio sobre los distintos algoritmos de aprendizaje que
implementan backpropagation y el ajuste de los correspondientes parámetros de estos:

\begin{itemize}[noitemsep]
	\item \textit{Learning rate} de los pesos.
	\item \textit{Learning rate} del bias.
	\item Criterio de parada: límite de epochs para el entrenamiento.
	\item Criterio de parada: límite de la función de rendimiento (goal)
	\item Criterio de parada: tiempo de aprendizaje.
\end{itemize}

La topología de la red se corresponderá con la configuración más óptima encontrada en la práctica P1\_Thyroid.

\section{Descripción del conjunto de datos}

El conjunto de datos utilizados se ha obtenido del dataset de ejemplo \emph{Thyroid} que provee Matlab. Los datos provienen del \emph{UCI Machine Learning Repository} \citep{Asuncion+Newman:2007} y fueron donados por la Universidad de California.

El dataset contiene datos de 7200 pacientes agrupados en dos matrices:

\begin{itemize}[noitemsep]
	\item $thyroidInputs$: matriz de 21x7200 con los datos de los 7200 pacientes caracterizados por 15 atributos binarios y 6 atributos continuos.
	\item $thyroidTargets$: matriz de 3x7200 en donde se asocia un vector de tres clases a cada paciente. En este vector se define a cuál de las tres clases pertenece el paciente.
\end{itemize}

Las tres clases que contiene el dataset son:

\begin{enumerate}[noitemsep]
	\item Paciente sano.
	\item Paciente con hipertiroidismo.
	\item Paciente con hipotiroidismo.
\end{enumerate}

\section{Descripción del procedimiento}

Para automatizar el estudio lo máximo posible, se ha realizado un script que realiza varios entrenamientos con los diferentes algoritmos de entrenamiento y variando los parámetros mencionados. 

Por cada combinación se ejecutan 20 experimentos y se toma el valor medio para reducir el impacto de la aleatoriedad en la inicialización de los pesos y bias.

Se han utilizado los siguientes algoritmos de entrenamiento:

\begin{itemize}[noitemsep]
	\item \lstinline|traincgb|: conjugate gradient backpropagation with Powell-Beale restarts. \citep{matlab:traincgb}
	\item \lstinline|traincgf|: conjugate gradient backpropagation with Fletcher-Reeves updates. \citep{matlab:traincgf}
\end{itemize}

Como se puede observar ambos utilizan el algoritmo de retropropagación para la actualización de los pesos. Y por lo tanto, son adecuados para el entrenamiento de un perceptrón multicapa.

Se modifican los siguientes parámetros para cada algoritmo:

\begin{itemize}[noitemsep]
	\item \lstinline|net.trainParam.max_fail|: maximum validation failures. Rango [10:10:100].
	\item \lstinline|net.trainParam.alpha|: scale factor that determines sufficient reduction in performance. Rango [0.001:0.001:0.004].
	\item \lstinline|net.trainParam.beta|: scale factor that determines sufficient large step size. Rango [0.1:0.2:0.8].
\end{itemize}

Como indicaderes generales se han empleado:

\begin{itemize}[noitemsep]
	\item Confusion value: fraction of samples misclassified.
	\item \lstinline|plotperform|: plot network performance. \citep{matlab:plotperform}
\end{itemize}

\section{Estudio}

TODO

\section{Conclusiones}

El rendimiento del algoritmo es muy sensible al ajuste del \textit{learning rate}. Si se establece un valor muy alto el algoritmo oscila y se vuelve inestable. Por el contrario, si se establece un valor muy bajo, la red neuronal puede demorarse demasiado en converger. 

TODO

% -------------------------------------------------------------
% Bibliography
% -------------------------------------------------------------
\bibliography{citations}
\bibliographystyle{plainnat}

\end{document}
