%%%%%%%%%%%%%%%%%%%%%%%%%%%%%%%%%%%%%%%%%%%%%%%%%%%%%%%%%%%%%%%
% P2 Self-organizing Map
% David Miguel Lozano
% Javier Martínez Riberas
% Universidad de Burgos - Octubre 2016
%%%%%%%%%%%%%%%%%%%%%%%%%%%%%%%%%%%%%%%%%%%%%%%%%%%%%%%%%%%%%%%

% -------------------------------------------------------------
% Preamble
% -------------------------------------------------------------
\documentclass[a4paper,12pt,titlepage]{article}

% -------------------------------------------------------------
% Packages
% -------------------------------------------------------------
\usepackage[utf8]{inputenc}	% Unicode support
\usepackage[T1]{fontenc}		% Font encoding
\usepackage[spanish]{babel}	% Languaje
\usepackage{lmodern}	% Typeface 
\usepackage{textcomp} % Special symbols
\usepackage{graphicx}	% Add pictures
\usepackage{pgfplots} % Graphs and charts
\usepackage{hyperref}	% Add a link to index entries
\usepackage{amsmath}	% Advanced math typesetting
\usepackage{amsfonts}	% Mathematical formulas
\usepackage{amssymb}	% Extended symbol collection
\usepackage{listings}	% Code formatting and highlighting
\usepackage{xcolor}		% Color package
\usepackage{enumitem}	% Customizing lists
\usepackage{parskip}	% Paragraph styles
\usepackage[a4paper]{geometry} 		% Margins
\usepackage[numbers,sort]{natbib}	% Bibliography management 

% -------------------------------------------------------------
% Configuration
% -------------------------------------------------------------
% Images path
\graphicspath{ {img/} }
% Graphs configuration 
\pgfplotsset{width=\textwidth,compat=1.9}
% Hyperlinks coloring
\hypersetup{
	colorlinks,
	linkcolor={green!40!black},
	citecolor={blue!50!black},
	urlcolor={blue!80!black}
}
% Define HRule
\newcommand{\HRule}[1]{\rule{\linewidth}{#1}}
% Define listings styles
\definecolor{codebg}{HTML}{EEEEEE}
\definecolor{codeframe}{HTML}{CCCCCC}
\definecolor{comments}{HTML}{009900}
\lstset{
  language=Matlab, 								% Programming language 
  backgroundcolor=\color{codebg},	% Background color
  frame=single, 									% Add frame around code
	framesep=10pt,									% Padding
	rulecolor=\color{codeframe},		% Don't change frame color
	upquote=true,										%
	breakatwhitespace=true,					% Break line only in spaces
	keepspaces=true,								% Keep indentation
	tabsize=2,											% Tab size
	title=\lstname, 								% Show filename as caption
	basicstyle=\ttfamily, 					% Size and font
  keywordstyle=\color{blue}\ttfamily,
	commentstyle=\color{comments},	% Color of comments
  morecomment=[l][\color{magenta}]{\#}
}
\begin{document}
% -------------------------------------------------------------
% Cover
% -------------------------------------------------------------
\author{David Miguel Lozano \ Javier Martínez Riberas}
\title{P2 Self-organizing Maps}
\date{08-10-2016}

\begin{titlepage}
	\centering
	\includegraphics[width=0.16\textwidth]{ubu-logo.png}\par
	\vspace{0.3cm}
	{\scshape\LARGE Universidad de Burgos \par}
	\vfill
	{\scshape\Large Computación Neuronal y Evolutiva \par}
	\HRule{2pt}
	{\huge\bfseries P2: Self-organizing Maps \par}
	\HRule{2pt}
	\\ [0.5cm]
	{Diseñar y entrenar un mapa auto-organizado con una rejilla bidimensional para el agrupamiento de un único conjunto de datos de muestra de tres dimensiones. Estudiando el efecto que tienen una serie de parámetros durante el entrenamiento de este.}
	\vfill
	Estudiantes:\par
	{\Large\scshape David Miguel Lozano \\ Javier Martínez Riberas \par}
	\vfill
	Profesor de la asignatura:\par
	\textsc{Álvaro Herrero Cosío}
	\vfill
	{\large 1º semestre 2016 \par}
\end{titlepage}

% -------------------------------------------------------------
% Contents
% -------------------------------------------------------------
\newpage
\tableofcontents
\begin{appendix}
  %\listoffigures
  %\listoftables
\end{appendix}

% -------------------------------------------------------------
% Body
% -------------------------------------------------------------
\newpage

\section{Introduction}

El objetivo de la práctica es diseñar y entrenar un mapa auto-organizado para una rejilla bidimensional con un único conjunto de datos de muestra de tres dimensiones. 

Se realizará un estudio sobre el impacto que supone variar los siguientes parámetros en el entrenamiento del mapa para el mismo conjunto de datos:

\begin{enumerate}[noitemsep]
	\item Dimensiones de la topología.
	\item Tamaño del vecindario.
	\item Función de topologia.
	\item Función de distancia entre neuronas.
\end{enumerate}

\section{Descripción del conjunto de datos}

El conjunto de datos utilizado para esta práctica ha sido generado aleatoriamente mediante la función \lstinline|nngen|. Se ha intentado conseguir un conjunto en donde se diferencien 7 clústeres próximos entre ellos pero que sean linealmente separables.

Los datos han sido generados de la siguiente manera:

\begin{lstlisting}
nngenc([0 5; 0 5; 0 5], 7, 200, 0.30);
\end{lstlisting}

El primer parámetro (\lstinline|bounds|) se corresponde con los rangos del espacio en donde generar los datos. Nuestros datos estarán en un espacio de tres dimensiones en el rago [0, 5] en cada dimensión.

El segundo parámetro (\lstinline|clusters|) es el número de clústeres a generar. Generaremos 7 clústeres linealmente diferenciables.

El tercer parámetro (\lstinline|points|) es el número de puntos por clúster. Nuestros clústeres tendrán 200 puntos cada uno.

Por último se indica la desviación estándar de los clústeres (\lstinline|std_dev|). Asignamos una desviación de un 30\% para que los clústeres no estén muy localizados y su agrupamiento no sea tan trivial.

Una vez generados los datos deseados, los hemos guardado en un fichero CSV para su posterior explotación en el estudio. 

\section{Descripción del procedimiento}

Para automatizar el estudio lo máximo posible, se ha realizado un script que entrena un mapa auto-organizado variando los parámetros de dimensiones de la topología, tamaño del vecindario, función de topología y función de distancia entre neuronas. 

Por cada configuración, se generan las siguientes figuras y se guardan como imágenes png:

\begin{itemize}[noitemsep]
	\item \lstinline|plotsomnd|: en esta representación se puede observar la topología del SOM (\textit{self-organizing map}), las conexiones directas entre neuronas y la distancia entre estas. Las neuronas se dibujan como hexágonos azules, las conexiones mediante líneas rojas y las distancias se representan mediante el coloreado de las parcelas, correspondiendo el color amarillo a distancias cortas y el negro a distancias grandes. \citep{matlab:plotsomnd}
	\item \lstinline|plotsomhits|: en esta representación se visualiza el número de datos para los que se está activando cada neurona (el número de vectores de entrada que clasifica). Se dibuja la topología del SOM y en cada parcela, correspondiente a una neurona, se muestra el número total así como un hexágono azul de tamaño relativo a este. \citep{matlab:plotsomhits}
	\item \lstinline|plotsompos|: muestra un gráfico en 2D en donde se visualizan la posición de las neuronas (puntos azules), las conexiones entre estas (líneas rojas) y las posiciones de los datos (puntos verdes). Permite visualizar cómo el SOM clasifica el espacio de entrada según evoluciona el entrenamiento. \citep{matlab:plotsompos}
\end{itemize}

Los comandos \lstinline|plotsomnd| y \lstinline|plotsomhits| están limitados a topologías \lstinline|hextop| o \lstinline|gridtop|, no pudiendo ser utilizados con \lstinline|randtop|. 

\emph{*Todo el código generado en esta práctica se encuentra disponible en el siguiente repositorio: 
\href{https://github.com/davidmigloz/neuronal-networks/tree/master/P2\_SelfOrganizingMap}{P2\_SelfOrganizingMap}.}

\section{Estudio}

TODO describir parámetros modificados (valores y rangos).

TODO se realizaron pruebas para acotar las configuraciones de los parámetros a estudiar.

\section{Conclusiones}

TODO



% -------------------------------------------------------------
% Bibliography
% -------------------------------------------------------------
\newpage
\bibliography{citations}
\bibliographystyle{plainnat}

\end{document}